\documentclass[11pt,twocolumn]{article}

\usepackage[margin=1in]{geometry}
\usepackage{graphicx}
\usepackage{amsmath}
\usepackage{amssymb}
\usepackage{booktabs}
\usepackage{multirow}
\usepackage{algorithm}
\usepackage{algorithmic}
\usepackage{url}
\usepackage{kotex}
\usepackage{setspace}

\title{\textbf{Semantic Entropy와 Semantic Energy의 상보적 결합을 통한 LLM 환각 탐지}\\[0.5em]
\large Hallucination Detection in LLMs via Complementary Combination of\\Semantic Entropy and Semantic Energy}

\author{문정혁\\
서울대학교 컴퓨터공학부\\
\texttt{student@snu.ac.kr}}

\date{2026년 2월}

\begin{document}

\maketitle

\begin{abstract}
대규모 언어모델(LLM)의 환각(hallucination)은 의료, 법률, 금융 분야에서 심각한 위험을 초래한다. 기존의 Semantic Entropy(SE) 기반 탐지 방법은 모델이 일관되게 틀린 답변을 생성하는 경우(Zero-SE 문제)에 한계를 보인다. 본 연구에서는 SE와 Semantic Energy의 상보적 특성을 분석하고, SE가 낮을 때 Energy로 대체하는 SE-gated cascade 방법을 제안한다. TruthfulQA 데이터셋에서 K=5 응답 샘플링 실험 결과, Zero-SE 영역(전체의 19\%)에서 Energy가 AUROC 0.736으로 환각을 구분할 수 있음을 확인하였다. 또한 SE-gated cascade는 SE 단독 대비 AUROC +0.030 개선(0.613에서 0.643)을 달성하였으며, SE와 Energy의 합집합 탐지율은 89.6\%에 도달하여 두 메트릭의 상보성을 검증하였다.
\end{abstract}

\noindent\textbf{Keywords:} LLM 환각 탐지, Semantic Entropy, Semantic Energy, Zero-SE 문제, Cascade Detection
\vspace{1em}

\section{서론}

\subsection{연구 배경}

대규모 언어모델(LLM)은 자연어 처리 분야에서 놀라운 성능을 보여주고 있으나, 사실과 다른 내용을 그럴듯하게 생성하는 환각(hallucination) 문제가 심각한 한계로 지적되고 있다. 특히 의료, 법률, 금융 등 정확성이 중요한 분야에서 LLM의 환각은 치명적인 결과를 초래할 수 있다.

환각 탐지를 위한 대표적인 방법으로 Semantic Entropy(SE)가 있다~\cite{farquhar2024}. SE는 하나의 질문에 대해 여러 응답을 샘플링한 뒤, 의미적으로 클러스터링하여 응답의 다양성을 측정한다. SE가 높으면 모델이 혼란스러워하는 것이므로 환각 가능성이 높다고 판단한다.

그러나 SE는 근본적인 한계를 가진다. 모델이 모든 응답에서 \textbf{일관되게 틀린 답변}을 생성하면 SE 값이 0에 가까워지고, 이 경우 SE만으로는 환각 여부를 판단할 수 없다. 우리는 이를 \textbf{Zero-SE 문제}라고 정의한다.

\subsection{연구 목적}

본 연구의 목적은 다음과 같다:
\begin{enumerate}
    \item Zero-SE 문제의 존재와 심각성을 정량적으로 분석한다.
    \item SE와 Semantic Energy가 서로 다른 유형의 환각을 탐지함을 검증한다.
    \item 두 메트릭을 상보적으로 결합하는 SE-gated cascade 방법을 제안한다.
\end{enumerate}

\section{관련 연구}

\subsection{Semantic Entropy (SE)}

Farquhar et al.~\cite{farquhar2024}이 Nature에서 제안한 Semantic Entropy는 LLM의 불확실성을 측정하는 대표적인 방법이다.

SE의 계산 과정은 다음과 같다:
\begin{enumerate}
    \item 하나의 질문에 대해 LLM으로부터 $K$개의 응답을 샘플링한다.
    \item 자연어 추론(NLI) 모델을 사용하여 응답들을 의미적으로 클러스터링한다.
    \item 클러스터 분포의 Shannon entropy를 계산한다.
\end{enumerate}

수식으로 표현하면:
\begin{equation}
    SE = -\sum_{c \in C} p(c) \log p(c)
\end{equation}
여기서 $C$는 의미 클러스터 집합이고, $p(c)$는 클러스터 $c$에 속하는 응답의 비율이다.

\subsection{Semantic Energy}

Ma et al.~\cite{ma2025}이 제안한 Semantic Energy는 토큰 단위의 확신도를 측정한다. Energy는 LLM이 각 토큰을 생성할 때 부여한 raw logit 값을 기반으로 계산된다:
\begin{equation}
    Energy = \frac{1}{nT} \sum_{i=1}^{n} \sum_{t=1}^{T_i} -z(x_t^{(i)})
\end{equation}
여기서 $z(x_t)$는 토큰 $x_t$의 logit 값이다.

SE는 응답 간 다양성(inter-response diversity)을 측정하고, Energy는 각 토큰의 내재적 확신도(intra-token confidence)를 측정한다. 따라서 SE=0인 상황(모든 응답이 동일)에서도 Energy는 여전히 정보를 제공할 수 있다.

\subsection{Zero-SE 문제}

Zero-SE는 $K$개 응답이 모두 단일 NLI 클러스터에 속하는 경우(SE$\approx$0)를 의미한다. SE는 정의상 0이므로 환각 여부에 대한 판별력이 전혀 없으나, 실제로 이 영역에 환각이 다수 포함되어 있다.

\section{제안 방법}

\subsection{핵심 통찰}

LLM이 환각을 일으킬 때, 모델이 해당 주제를 아는지 여부에 따라 근본적으로 다른 방식으로 환각한다:

\begin{table}[htbp]
\centering
\caption{환각 유형별 특성과 최적 탐지기}
\label{tab:hallucination_types}
\begin{tabular}{@{}llll@{}}
\toprule
유형 & 원인 & 행동적 신호 & 최적 탐지기 \\
\midrule
혼란 & 알지만 헷갈림 & 다양한 오답 & SE \\
지어냄 & 모르고 지어냄 & 일관된 오답 & Energy \\
\bottomrule
\end{tabular}
\end{table}

\subsection{SE-Gated Cascade}

제안하는 SE-gated cascade는 그림~\ref{fig:cascade_diagram}과 같다.

\begin{figure}[htbp]
    \centering
    \includegraphics[width=0.95\columnwidth]{images/fig6_story_diagram.png}
    \caption{SE-Gated Cascade 개념도}
    \label{fig:cascade_diagram}
\end{figure}

\begin{algorithm}[htbp]
\caption{SE-Gated Cascade Detection}
\label{alg:cascade}
\begin{algorithmic}[1]
\REQUIRE 질문 $q$, 임계값 $\tau$
\STATE LLM으로 $K=5$ 응답 생성
\STATE NLI 클러스터링으로 SE 계산
\STATE 토큰 logit으로 Energy 계산
\IF{$SE < \tau$}
    \STATE \textbf{return} Energy (지어냄 영역)
\ELSE
    \STATE \textbf{return} SE (혼란 영역)
\ENDIF
\end{algorithmic}
\end{algorithm}

본 연구에서는 TruthfulQA 데이터셋에서 검증 세트를 통해 최적 임계값 $\tau=0.526$을 도출하였다.

\section{실험}

\subsection{실험 설정}

TruthfulQA 데이터셋~\cite{truthfulqa}을 사용하였다. 이 데이터셋은 대중적 오개념을 유도하는 질문들로 구성되어 있다.

\begin{table}[htbp]
\centering
\caption{실험 파이프라인 설정}
\label{tab:pipeline}
\begin{tabular}{@{}ll@{}}
\toprule
항목 & 설정 \\
\midrule
LLM & Qwen2.5-3B-Instruct \\
NLI 모델 & DeBERTa-large-mnli \\
샘플링 수 ($K$) & 5 \\
Temperature & 0.7 \\
전체 샘플 & 200 (환각 164, 정상 36) \\
\bottomrule
\end{tabular}
\end{table}

\subsection{Zero-SE 현상 분석}

\begin{figure}[htbp]
    \centering
    \includegraphics[width=0.95\columnwidth]{images/fig1_zero_se_overview.png}
    \caption{Zero-SE 현상 개요}
    \label{fig:zero_se_overview}
\end{figure}

TruthfulQA에서 Zero-SE 현상을 정량적으로 분석하였다 (표~\ref{tab:zero_se}).

\begin{table}[htbp]
\centering
\caption{TruthfulQA Zero-SE 영역 분석}
\label{tab:zero_se}
\begin{tabular}{@{}lr@{}}
\toprule
지표 & 값 \\
\midrule
Zero-SE 비율 & 19.0\% (38/200) \\
Zero-SE 내 환각률 & 73.7\% (28/38) \\
Zero-SE Energy AUROC & 0.736 \\
95\% 신뢰구간 & [0.52, 0.93] \\
\bottomrule
\end{tabular}
\end{table}

전체 샘플의 19\%가 Zero-SE에 해당하며, 이 중 73.7\%가 실제 환각이다. SE로는 판별 불가하지만, Energy AUROC 0.736으로 구분 가능하다.

\subsection{SE 구간별 Crossover 분석}

\begin{figure}[htbp]
    \centering
    \includegraphics[width=0.95\columnwidth]{images/fig2_se_bin_crossover.png}
    \caption{SE 구간별 SE vs Energy AUROC 비교}
    \label{fig:crossover}
\end{figure}

표~\ref{tab:crossover}에서 Zero-SE 영역에서는 Energy가 우세하고, High-SE 영역에서는 SE가 우세함을 확인할 수 있다.

\begin{table}[htbp]
\centering
\caption{SE 구간별 탐지 성능}
\label{tab:crossover}
\begin{tabular}{@{}lrrr@{}}
\toprule
SE 구간 & n & SE & Energy \\
\midrule
Zero [0, 0.05] & 38 & N/A & \textbf{0.736} \\
High (0.6, 1.0] & 44 & \textbf{0.664} & 0.517 \\
Very High (1.0+) & 95 & \textbf{0.658} & 0.422 \\
\bottomrule
\end{tabular}
\end{table}

\subsection{SE-Gated Cascade 성능}

\begin{figure}[htbp]
    \centering
    \includegraphics[width=0.95\columnwidth]{images/fig4_cascade_sweep.png}
    \caption{Cascade 임계값 sweep 결과}
    \label{fig:cascade_sweep}
\end{figure}

\begin{table}[htbp]
\centering
\caption{탐지 방법별 AUROC 비교}
\label{tab:cascade_result}
\begin{tabular}{@{}lrr@{}}
\toprule
방법 & AUROC & $\Delta$ \\
\midrule
SE-only & 0.613 & - \\
Energy-only & 0.550 & -0.063 \\
\textbf{Cascade} & \textbf{0.643} & \textbf{+0.030} \\
\bottomrule
\end{tabular}
\end{table}

Cascade 방법이 SE-only 대비 AUROC 0.030 개선을 달성하였다.

\subsection{상보성 분석}

\begin{figure}[htbp]
    \centering
    \includegraphics[width=0.95\columnwidth]{images/fig3_complementarity.png}
    \caption{SE와 Energy의 환각 탐지 영역 비교}
    \label{fig:complementarity}
\end{figure}

\begin{table}[htbp]
\centering
\caption{환각 탐지 상보성 (164개 환각 기준)}
\label{tab:complementarity}
\begin{tabular}{@{}lr@{}}
\toprule
탐지 영역 & 비율 \\
\midrule
SE만 탐지 & 9.8\% (16개) \\
Energy만 탐지 & 17.7\% (29개) \\
둘 다 탐지 & 62.2\% (102개) \\
둘 다 실패 & 10.4\% (17개) \\
\midrule
\textbf{합집합 탐지율} & \textbf{89.6\%} \\
\bottomrule
\end{tabular}
\end{table}

Energy만 탐지하는 17.7\%는 SE로는 잡을 수 없는 환각이다.

\subsection{전체 비교}

\begin{figure}[htbp]
    \centering
    \includegraphics[width=0.95\columnwidth]{images/fig5_overall_comparison.png}
    \caption{SE, Energy, Cascade 전체 성능 비교}
    \label{fig:overall}
\end{figure}

\section{결론}

본 연구에서는 LLM 환각 탐지에서 SE의 한계인 Zero-SE 문제를 정의하고, Semantic Energy와의 상보적 결합을 통해 이를 해결하는 방법을 제안하였다.

TruthfulQA 데이터셋에서의 실험을 통해 다음을 확인하였다:
\begin{enumerate}
    \item \textbf{Zero-SE 문제}: 전체 샘플의 19\%가 Zero-SE에 해당하며, 이 중 73.7\%가 환각이다.
    \item \textbf{Energy의 효과}: Zero-SE 영역에서 Energy가 AUROC 0.736으로 환각을 구분한다.
    \item \textbf{Cascade의 개선}: SE-gated cascade는 SE-only 대비 AUROC +0.030 개선을 달성한다.
    \item \textbf{상보성}: 합집합 탐지율은 89.6\%에 도달한다.
\end{enumerate}

\subsection{향후 연구}

다양한 데이터셋 검증, 적응적 임계값 결정, Energy의 이론적 분석, Cross-dataset 일반화 등이 향후 연구 과제이다.

\begin{thebibliography}{99}
\bibitem{farquhar2024} S. Farquhar et al., ``Detecting hallucinations in large language models using semantic entropy,'' \textit{Nature}, vol. 630, pp. 625--630, 2024.
\bibitem{ma2025} Z. Ma et al., ``Semantic Energy: A novel approach for detecting confabulation in language models,'' \textit{arXiv}, 2025.
\bibitem{truthfulqa} S. Lin et al., ``TruthfulQA: Measuring how models mimic human falsehoods,'' \textit{ACL}, 2022.
\end{thebibliography}

\end{document}
