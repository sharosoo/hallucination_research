% TikZ 다이어그램 코드 모음
% main.tex에 % TikZ 다이어그램 - 학술 논문 스타일
% 색상 팔레트 정의
\definecolor{primaryblue}{RGB}{31,119,180}
\definecolor{secondaryorange}{RGB}{255,127,14}
\definecolor{accentgreen}{RGB}{44,160,44}
\definecolor{accentpurple}{RGB}{148,103,189}
\definecolor{neutralgray}{RGB}{127,127,127}
\definecolor{lightgray}{RGB}{200,200,200}
\definecolor{hallred}{RGB}{214,39,40}
\definecolor{normgreen}{RGB}{44,160,44}

%=============================================================================
% Figure 1: SE-Gated Cascade 개념도 (깔끔한 플로우차트)
%=============================================================================
\newcommand{\cascadediagram}{
\begin{tikzpicture}[
    node distance=1.5cm,
    >={Stealth[length=2.5mm]},
    box/.style={
        rectangle, draw=black!70, line width=0.8pt,
        rounded corners=2pt, minimum width=3cm, minimum height=0.9cm,
        align=center, font=\small
    },
    inputbox/.style={box, fill=lightgray!30},
    processbox/.style={box, fill=primaryblue!15},
    energybox/.style={box, fill=secondaryorange!20},
    outputbox/.style={box, fill=accentgreen!20},
    decision/.style={
        diamond, draw=black!70, line width=0.8pt,
        aspect=2.5, minimum width=2cm, align=center, font=\small,
        fill=accentpurple!15
    },
    arrow/.style={->, line width=0.8pt, black!70},
    label/.style={font=\scriptsize, text=black!60}
]

% 상단: 입력 → 샘플링 → 클러스터링 (가로)
\node[inputbox] (input) {Input: 질문 $q$};
\node[processbox, right=2cm of input] (sample) {LLM 샘플링 ($K$=5)};
\node[processbox, right=2cm of sample] (cluster) {NLI 클러스터링};

% 중단: 메트릭 계산 (병렬)
\node[processbox, below left=1.5cm and 0.3cm of cluster] (se) {SE 계산};
\node[energybox, below right=1.5cm and 0.3cm of cluster] (energy) {Energy 계산};

% 하단: 분기 및 출력
\node[decision, below=2.5cm of cluster] (gate) {$|C|=1$?};
\node[outputbox, below=1.8cm of gate] (output) {환각 점수};

% 화살표
\draw[arrow] (input) -- (sample);
\draw[arrow] (sample) -- (cluster);
\draw[arrow] (cluster) -- (se);
\draw[arrow] (cluster) -- (energy);
\draw[arrow] (se) -- (gate);
\draw[arrow] (energy) -- (gate);
\draw[arrow] (gate) -- node[right, label] {Energy (Yes) / SE (No)} (output);

% 클러스터 수 표시
\node[font=\scriptsize, text=black!50, right=0.1cm of cluster] {$\rightarrow |C|$};

\end{tikzpicture}
}

%=============================================================================
% Figure 2: Zero-SE 현상 요약 (3-panel) - 완전 정렬 버전
%=============================================================================
\newcommand{\zerosefigure}{
\centering
\begin{tikzpicture}[font=\small]

% 모든 패널: y=0 ~ y=2.5 범위, 제목은 y=-0.5

% Panel A: 파이 차트 (하단을 y=0에 맞춤, 중심=radius)
\begin{scope}
    \def\radius{1.1}
    \def\centery{1.1}  % 중심 y = radius이면 하단이 y=0
    \fill[primaryblue!50] (0,\centery) circle (\radius);
    \fill[hallred!60] (0,\centery) -- ++(90:\radius) arc (90:21.6:\radius) -- cycle;
    
    \node[font=\scriptsize, white] at ($(0,\centery)+(55:0.7)$) {\textbf{19\%}};
    \node[font=\scriptsize, white] at ($(0,\centery)+(-70:0.6)$) {\textbf{81\%}};
    
    % 범례 (막대 차트 높이에 맞춤)
    \fill[hallred!60] (1.5,1.8) rectangle (1.75,2.0);
    \node[font=\tiny, anchor=west] at (1.8, 1.9) {Zero-SE};
    \fill[primaryblue!50] (1.5,1.4) rectangle (1.75,1.6);
    \node[font=\tiny, anchor=west] at (1.8, 1.5) {Non-Zero};
\end{scope}
\node[font=\scriptsize\bfseries] at (0,-0.5) {(a) Zero-SE 비율};
\node[font=\tiny, text=black!60] at (0,-0.85) {n=200};

% Panel B: 환각/정상 막대 (y=0 ~ 2.5)
\begin{scope}[xshift=5cm]
    \draw[->, black!70] (0,0) -- (0,2.6);
    \draw[->, black!70] (0,0) -- (2.5,0);
    
    \fill[hallred!60] (0.3,0) rectangle (0.95,2.15);
    \node[font=\scriptsize] at (0.625,2.3) {\textbf{28}};
    
    \fill[normgreen!60] (1.3,0) rectangle (1.95,0.77);
    \node[font=\scriptsize] at (1.625,0.92) {\textbf{10}};
    
    \node[font=\tiny] at (0.625,-0.2) {환각};
    \node[font=\tiny] at (1.625,-0.2) {정상};
    
    \node[font=\tiny, anchor=east] at (-0.05,0) {0};
    \node[font=\tiny, anchor=east] at (-0.05,2.15) {28};
\end{scope}
\node[font=\scriptsize\bfseries] at (6.1,-0.5) {(b) Zero-SE 구성};
\node[font=\tiny, text=black!60] at (6.1,-0.85) {환각률: 73.7\%};

% Panel C: Energy AUROC (y=0 ~ 2.5)
\begin{scope}[xshift=9.2cm]
    \draw[->, black!70] (0,0) -- (0,2.6);
    \draw[->, black!70] (0,0) -- (2,0);
    
    \draw[dashed, black!40, line width=0.5pt] (0,1.25) -- (1.7,1.25);
    \node[font=\tiny, text=black!40, anchor=west] at (1.4,1.35) {0.5};
    
    \fill[secondaryorange!70] (0.4,0) rectangle (1.2,1.84);
    \node[font=\scriptsize] at (0.8,2.0) {\textbf{0.736}};
    
    \node[font=\tiny] at (0.8,-0.2) {Energy};
    
    \node[font=\tiny, anchor=east] at (-0.05,0) {0};
    \node[font=\tiny, anchor=east] at (-0.05,2.5) {1.0};
\end{scope}
\node[font=\scriptsize\bfseries] at (10,-0.5) {(c) Zero-SE 탐지};

\end{tikzpicture}
}

%=============================================================================
% Figure 3: SE vs Energy Crossover (Grouped Bar)
%=============================================================================
\newcommand{\crossoverfigure}{
\begin{tikzpicture}
\begin{axis}[
    ybar,
    width=11cm, height=6.5cm,
    bar width=14pt,
    ylabel={AUROC},
    ylabel style={font=\small},
    ymin=0, ymax=0.9,
    xtick={1,2,3},
    xticklabels={Zero-SE\\{\tiny $|C|=1$}, Medium\\{\tiny $0.5<\text{SE}\leq1$}, High\\{\tiny $\text{SE}>1$}},
    xticklabel style={align=center, font=\small},
    tick label style={font=\small},
    legend style={
        at={(0.98,0.98)}, anchor=north east, 
        font=\small, 
        draw=none, fill=white, fill opacity=0.8,
        legend columns=1
    },
    nodes near coords,
    nodes near coords style={font=\scriptsize, /pgf/number format/.cd, fixed, precision=2},
    axis lines=left,
    enlarge x limits=0.25,
    extra y ticks={0.5},
    extra y tick style={grid=major, grid style={dashed, lightgray, line width=0.5pt}},
    extra y tick labels={},
    ymajorgrids=false,
]
\addplot[fill=primaryblue!70, draw=primaryblue] coordinates {(1,0.001) (2,0.61) (3,0.66)};
\addplot[fill=secondaryorange!70, draw=secondaryorange] coordinates {(1,0.74) (2,0.52) (3,0.42)};
\legend{SE, Energy}
\end{axis}

% 주석
% (요청 반영) 우세 방향 화살표/주석 제거

\end{tikzpicture}
}

%=============================================================================
% Figure 4: 상보성 분석 - 단순 막대 그래프
%=============================================================================
\newcommand{\complementarityfigure}{
\centering
\begin{tikzpicture}[font=\small]
% 축
\draw[->] (0,0) -- (0,5.5);
\draw[->] (0,0) -- (10,0);

% Y축 라벨 및 눈금
\node[font=\small, rotate=90, anchor=south] at (-0.6,2.75) {환각 샘플 수};
\foreach \y/\val in {0/0, 1.08/20, 2.16/40, 3.24/60, 4.32/80, 5.4/108} {
    \draw (-0.1,\y) -- (0.1,\y);
}
\node[font=\tiny, left] at (0,0) {0};
\node[font=\tiny, left] at (0,2.7) {50};
\node[font=\tiny, left] at (0,5.4) {108};

% 막대 1: SE만 (22)
\fill[primaryblue!70] (0.8,0) rectangle (2,1.1);
\node[font=\scriptsize] at (1.4,1.3) {22};
\node[font=\tiny, text=black!60] at (1.4,1.6) {(13.4\%)};
\node[font=\scriptsize, align=center] at (1.4,-0.4) {SE만\\탐지};

% 막대 2: Energy만 (22)
\fill[secondaryorange!70] (2.8,0) rectangle (4,1.1);
\node[font=\scriptsize] at (3.4,1.3) {22};
\node[font=\tiny, text=black!60] at (3.4,1.6) {(13.4\%)};
\node[font=\scriptsize, align=center] at (3.4,-0.4) {Energy만\\탐지};

% 막대 3: 둘 다 (12)
\fill[accentpurple!60] (4.8,0) rectangle (6,0.6);
\node[font=\scriptsize] at (5.4,0.8) {12};
\node[font=\tiny, text=black!60] at (5.4,1.1) {(7.3\%)};
\node[font=\scriptsize, align=center] at (5.4,-0.4) {둘 다\\탐지};

% 막대 4: 미탐지 (108)
\fill[neutralgray!50] (6.8,0) rectangle (8,5.4);
\node[font=\scriptsize] at (7.4,5.6) {108};
\node[font=\tiny, text=black!60] at (7.4,5.9) {(65.9\%)};
\node[font=\scriptsize, align=center] at (7.4,-0.4) {미탐지};

% 합집합 강조 박스
\draw[rounded corners=3pt, fill=accentgreen!15, draw=accentgreen!60, line width=1pt] 
    (0.5, 6.3) rectangle (5.5, 7.0);
\node[font=\small] at (3, 6.65) {합집합 탐지율: \textbf{34.1\%}};

% 제목
\node[font=\small\bfseries] at (4.5, 7.5) {상보성 분석 (n=164 환각)};

\end{tikzpicture}
}

%=============================================================================
% Figure 5: 전체 성능 비교 (Simple Bar)
%=============================================================================
\newcommand{\overallfigure}{
\begin{tikzpicture}
\begin{axis}[
    ybar,
    width=9cm, height=6cm,
    bar width=28pt,
    ylabel={AUROC},
    ylabel style={font=\small},
    ymin=0.45, ymax=0.70,
    xtick={1,2,3},
    xticklabels={SE-only, Energy-only, Cascade},
    xticklabel style={font=\small},
    tick label style={font=\small},
    nodes near coords,
    nodes near coords style={font=\small, /pgf/number format/.cd, fixed, precision=3},
    axis lines=left,
    enlarge x limits=0.35,
    extra y ticks={0.5},
    extra y tick style={grid=major, grid style={dashed, lightgray}},
    extra y tick labels={},
]
\addplot[fill=primaryblue!70, draw=primaryblue, bar shift=0pt] coordinates {(1,0.613)};
\addplot[fill=secondaryorange!70, draw=secondaryorange, bar shift=0pt] coordinates {(2,0.550)};
\addplot[fill=accentgreen!70, draw=accentgreen, bar shift=0pt] coordinates {(3,0.642)};
\end{axis}

% 개선 화살표
\draw[->, line width=1pt, accentgreen!70!black] (6.8, 4.2) -- (6.8, 5.0);
\node[font=\scriptsize, text=accentgreen!70!black, anchor=south] at (6.8, 5.1) {\textbf{+0.029}};

% Cascade 설명
\node[font=\tiny, text=black!60, align=center, anchor=north] at (6.8, 0.8) {$|C|=1 \rightarrow$ Energy\\otherwise $\rightarrow$ SE};

\end{tikzpicture}
}
로 포함

% 필요한 패키지 (main.tex의 preamble에 추가)
% \usepackage{tikz}
% \usetikzlibrary{shapes.geometric, arrows.meta, positioning, calc, backgrounds}
% \usepackage{pgfplots}
% \pgfplotsset{compat=1.18}

%=============================================================================
% Figure 1: SE-Gated Cascade 개념도 (클러스터 기반)
%=============================================================================
\newcommand{\cascadediagram}{
\begin{tikzpicture}[
    node distance=1.2cm and 1.5cm,
    >={Stealth[length=3mm]},
    box/.style={rectangle, draw, rounded corners=3pt, minimum width=2.8cm, minimum height=1cm, align=center, font=\small},
    decision/.style={diamond, draw, aspect=2, minimum width=2.5cm, minimum height=1.2cm, align=center, font=\small},
    input/.style={box, fill=yellow!20},
    process/.style={box, fill=blue!15},
    energy/.style={box, fill=orange!25},
    output/.style={box, fill=green!20},
    result/.style={box, fill=purple!15},
    arrow/.style={->, thick},
    label/.style={font=\footnotesize, text=gray}
]

% 노드 배치
\node[input] (q) {질문 $q$};
\node[process, right=of q] (llm) {LLM 샘플링\\(K=5 응답)};
\node[process, below=of llm] (nli) {NLI 클러스터링\\클러스터 집합 $C$};

% 메트릭 계산
\node[process, below left=1cm and 0.5cm of nli] (se) {SE 계산\\$-\sum p(c)\log p(c)$};
\node[energy, below right=1cm and 0.5cm of nli] (energy) {Energy 계산\\토큰 logit 기반};

% 분기점
\node[decision, below=2cm of nli] (gate) {$|C| = 1$?};

% 결과
\node[result, below left=1.2cm and 1cm of gate] (use_energy) {Energy 사용\\(Zero-SE)};
\node[result, below right=1.2cm and 1cm of gate] (use_se) {SE 사용\\($|C| \geq 2$)};

% 최종 출력
\node[output, below=3.5cm of gate] (final) {환각 점수};

% 화살표
\draw[arrow] (q) -- (llm);
\draw[arrow] (llm) -- (nli);
\draw[arrow] (nli) -- (se);
\draw[arrow] (nli) -- (energy);
\draw[arrow] (se) -- (gate);
\draw[arrow] (energy) -- (gate);
\draw[arrow] (gate) -- node[left, label] {Yes} (use_energy);
\draw[arrow] (gate) -- node[right, label] {No} (use_se);
\draw[arrow] (use_energy) -- (final);
\draw[arrow] (use_se) -- (final);

% 배경 영역 표시
\begin{scope}[on background layer]
    \node[draw=blue!50, dashed, rounded corners, fit=(se)(energy)(gate), inner sep=8pt, label={[font=\footnotesize, text=blue!70]above:Cascade Gate}] {};
\end{scope}

\end{tikzpicture}
}

%=============================================================================
% Figure 2: Zero-SE 현상 요약 (3개 서브플롯)
%=============================================================================
\newcommand{\zerosefigure}{
\begin{tikzpicture}
\begin{scope}[local bounding box=pie]
% 파이 차트: Zero-SE 비율
\def\zeroangle{68.4} % 19% = 68.4도
\fill[cyan!60] (0,0) -- (90:1.5) arc (90:90-\zeroangle:1.5) -- cycle;
\fill[red!60] (0,0) -- (90-\zeroangle:1.5) arc (90-\zeroangle:90-360:1.5) -- cycle;
\draw (0,0) circle (1.5);
\node[font=\footnotesize] at (90-\zeroangle/2:1) {19\%};
\node[font=\footnotesize] at (-90:0.8) {81\%};
\node[below, font=\small\bfseries] at (0,-1.8) {Zero-SE 비율};
\node[font=\tiny, red!70] at (1.8, 0.5) {Zero-SE};
\node[font=\tiny, cyan!70] at (1.8, -0.5) {Non-Zero};
\end{scope}

\begin{scope}[xshift=5cm]
% 막대 그래프: Zero-SE 구성
\draw[->] (0,0) -- (0,2.5) node[left, font=\tiny, rotate=90, anchor=south] {Count};
\draw[->] (0,0) -- (3,0);
\fill[red!60] (0.3,0) rectangle (1.2,2.1); % 28
\fill[green!60] (1.5,0) rectangle (2.4,0.75); % 10
\node[font=\tiny] at (0.75, 2.3) {28};
\node[font=\tiny] at (1.95, 0.95) {10};
\node[font=\tiny] at (0.75, -0.3) {환각};
\node[font=\tiny] at (1.95, -0.3) {정상};
\node[below, font=\small\bfseries] at (1.35,-0.7) {Zero-SE 구성};
\node[font=\tiny, text=gray] at (1.35, 2.8) {환각률: 73.7\%};
\end{scope}

\begin{scope}[xshift=9.5cm]
% 막대 그래프: Energy AUROC
\draw[->] (0,0) -- (0,2.5) node[left, font=\tiny, rotate=90, anchor=south] {AUROC};
\draw[->] (0,0) -- (2.5,0);
\draw[dashed, gray] (0,1.25) -- (2.3,1.25) node[right, font=\tiny] {0.5};
\fill[blue!60] (0.5,0) rectangle (1.8,1.84); % 0.736
\node[font=\tiny] at (1.15, 2.0) {0.736};
\node[font=\tiny] at (1.15, -0.3) {Energy};
\node[below, font=\small\bfseries] at (1.15,-0.7) {Zero-SE 탐지};
% Y축 눈금
\node[font=\tiny, left] at (0, 0) {0};
\node[font=\tiny, left] at (0, 1.25) {0.5};
\node[font=\tiny, left] at (0, 2.5) {1.0};
\end{scope}

\end{tikzpicture}
}

%=============================================================================
% Figure 3: SE vs Energy Crossover (구간별 AUROC)
%=============================================================================
\newcommand{\crossoverfigure}{
\begin{tikzpicture}
\begin{axis}[
    ybar,
    width=12cm,
    height=7cm,
    bar width=18pt,
    ylabel={AUROC},
    xlabel={SE 구간},
    ymin=0, ymax=1,
    xtick={1,2,3},
    xticklabels={Zero, Medium, High},
    legend style={at={(0.98,0.98)}, anchor=north east, font=\small},
    nodes near coords,
    nodes near coords style={font=\footnotesize},
    every axis plot/.append style={fill opacity=0.8},
    extra y ticks={0.5},
    extra y tick style={grid=major, grid style={dashed, gray}},
    extra y tick labels={},
]
\addplot[fill=blue!60] coordinates {(1, 0) (2, 0.61) (3, 0.66)};
\addplot[fill=red!60] coordinates {(1, 0.74) (2, 0.52) (3, 0.42)};
\legend{SE AUROC, Energy AUROC}
\end{axis}

% 구간 설명
\node[font=\tiny, text=gray] at (2.2, -0.3) {$[0, 0.1]$};
\node[font=\tiny, text=gray] at (5.5, -0.3) {$(0.5, 1.0]$};
\node[font=\tiny, text=gray] at (9.2, -0.3) {$(1.0+)$};

% Crossover 화살표 및 설명
\node[font=\footnotesize, text=red!70] at (2.5, 6.8) {Energy 우세};
\node[font=\footnotesize, text=blue!70] at (9.5, 6.8) {SE 우세};

\end{tikzpicture}
}

%=============================================================================
% Figure 4: 상보성 분석 (Venn-style 또는 Bar)
%=============================================================================
\newcommand{\complementarityfigure}{
\begin{tikzpicture}
\begin{axis}[
    ybar,
    width=11cm,
    height=7cm,
    bar width=25pt,
    ylabel={환각 샘플 수},
    ymin=0, ymax=120,
    xtick={1,2,3,4},
    xticklabels={SE만, Energy만, 둘 다, 미탐지},
    nodes near coords,
    nodes near coords style={font=\small, anchor=south},
    every axis plot/.append style={fill opacity=0.85},
    title={상보성 분석 (n=164 환각)},
    title style={font=\bfseries},
]
\addplot[fill=blue!60] coordinates {(1, 22)};
\addplot[fill=orange!60] coordinates {(2, 22)};
\addplot[fill=purple!60] coordinates {(3, 12)};
\addplot[fill=gray!50] coordinates {(4, 108)};
\end{axis}

% 비율 표시
\node[font=\footnotesize, text=gray] at (1.8, 2.5) {13.4\%};
\node[font=\footnotesize, text=gray] at (4.3, 2.5) {13.4\%};
\node[font=\footnotesize, text=gray] at (6.8, 1.8) {7.3\%};
\node[font=\footnotesize, text=gray] at (9.3, 5.5) {65.9\%};

% 합집합 탐지율 강조
\node[draw, rounded corners, fill=green!10, font=\small] at (5.5, 7.5) {합집합 탐지율: \textbf{34.1\%}};

\end{tikzpicture}
}

%=============================================================================
% Figure 5: 전체 성능 비교
%=============================================================================
\newcommand{\overallfigure}{
\begin{tikzpicture}
\begin{axis}[
    ybar,
    width=10cm,
    height=7cm,
    bar width=30pt,
    ylabel={AUROC},
    ymin=0.4, ymax=0.75,
    xtick={1,2,3},
    xticklabels={SE-only, Energy-only, Cascade},
    nodes near coords,
    nodes near coords style={font=\small, anchor=south},
    every axis plot/.append style={fill opacity=0.85},
    title={전체 성능 비교 (n=200)},
    title style={font=\bfseries},
    extra y ticks={0.5},
    extra y tick style={grid=major, grid style={dashed, gray}},
    extra y tick labels={},
]
\addplot[fill=blue!60] coordinates {(1, 0.613)};
\addplot[fill=orange!60] coordinates {(2, 0.550)};
\addplot[fill=purple!70] coordinates {(3, 0.642)};
\end{axis}

% Cascade 설명
\node[font=\footnotesize, text=purple!70, align=center] at (8, 1.5) {$|C|=1 \rightarrow$ Energy\\그 외 $\rightarrow$ SE};

% 개선 표시
\draw[->, thick, green!60!black] (7.3, 5.8) -- (7.3, 6.8);
\node[font=\footnotesize, text=green!60!black] at (7.3, 7.2) {+0.029};

\end{tikzpicture}
}
